\chapter{Einleitung}

Das folgende Kapitel dient der Einführung in die Problemstellung, Motivation sowie Ziele und Vorgehensweisen der vorliegenden Arbeit.

\section{Problemstellung und Motivation}

Die steigende Menge an Binärdaten im Kontext von Web Services, die in verschiedenen Anwendungen generiert werden, stellt eine große Herausforderung dar. Dabei ist es von großer Bedeutung, dass diese Daten sicher, zuverlässig und schnell gespeichert und abgerufen werden können. Vor diesem Hintergrund stellen sich Fragen nach der Auswahl eines geeigneten Speichersystems, das die Anforderungen wie Performance, Verfügbarkeit, Sicherheit und API-Anbindung erfüllt.  Zudem müssen Mechanismen bereitgestellt werden, um den Zugriff auf die Daten zu beschränken durch sichere, zeitlich begrenzte URL’s.\\

Diese Bachelorarbeit richtet sich auf das Problem einer Full-Service-Ticketing Software „leoticket“, die vom Unternehmen Leomedia GmbH entwickelt wurde. 
Leomedia GmbH ist ein Unternehmen, das Software für Medienunternehmen wie Zeitungsverlage, Radiosender, Veranstalter, Künstler und Kulturvereine entwickelt. \textcite{leomedia-web}\\
Leoticket ist eines der vielen Produkte von Leomedia, dass Services wie Online-Kartenvorverkäufe, Abendkassen, den Einlass bei der Veranstaltung, Statistiken und Abrechnungen und die Planung der Veranstaltung realisiert.\textcite{leomedia-web}\\ 

% Kontext des Problems beschreiben

Das Problem liegt bei der Speicherung und Bereitstellung der Daten. Da es sich um Replikationen der Daten handelt, ist die Belastung des Systems hoch. Hohe Daten werden herumgeschoben. Die Bandbreite ist bei der Übertragung begrenzt. Ein weiteres Problem ist die Bereitstellung der Daten über Email Anhänge. Anhänge dürfen eine bestimmte Speichergröße nicht überschreiten. Wenn Ticketkäufer zehn oder 100 digitale Tickets kaufen, dann müssen diese über Email Anhänge bereitgestellt werden.\\


\section{Zieldefinition und Vorgehensweise}

Ziel dieser Arbeit ist die Realisierung eines Prototypen anhand der ausgewählten Speicherlösung die Binärdaten durch sichere, zeitlich begrenzte URLs bereitstellt. Dabei werden folgenden Fragen gestellt:

\begin{itemize}
	\item Welches Speichersystem ist im Hinblick auf Kosten, Performance und Verfügbarkeit für die Persistenz von Binärdaten besonders geeignet? 
	\item Wie kann man Daten durch sichere, zeitlich begrenzte URL's bereitstellen?
\end{itemize}

Im Rahmen der vorliegenden Bachelorarbeit werden verschiedene Arten von Speichersystemen untersucht, um die Forschungsfragen zu beantworten. Dabei erfolgt eine Analyse der aktuell verfügbaren Speichertechnologien auf dem Markt hinsichtlich ihrer Eigenschaften wie Sicherheit, Verfügbarkeit, Performance und Kosten. Bei der Berücksichtigung der Integration des Speichersystems in Software-Produkte wird auch die API-Anbindung des Speichersystems betrachtet. Zur sicheren Bereitstellung von Dateien werden zudem geeignete Cloud-Provider miteinander verglichen. Kosten- und Performance-Kalkulationen werden durchgeführt, um eine geeignete Speicherlösung auszuwählen. Die Ergebnisse werden anschließend ausgewertet.\\

Im Zuge der prototypischen Umsetzung werden die ausgewählten Technologien implementiert und Testdateien zur Verfügung gestellt. Nach der Durchführung von Messungen zur Performance auf Testdaten erfolgt eine Zusammenfassung der Implementierung.\\

Abschließend werden die Ergebnisse nochmals dargestellt und interpretiert sowie Schwächen und Grenzen des Prototyps aufgezeigt. Zur Einhaltung des roten Fadens der Arbeit werden die gestellten Forschungsfragen beantwortet und potenzielle Anwendungen des Prototyps aufgelistet.

