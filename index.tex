%*****************************************						
% gi004@hdm-stuttgart.de														
%******************************************
%
% Masterdatei für die einzelnen Kapitel
%********************************************
% Allgemeine Einstellungen
%********************************************
\documentclass[
               	a4paper, 
                oneside, 
                fontsize=12pt,
                toc=bibliography]
                {report}
\usepackage[utf8]{inputenc}
\usepackage[ngerman]{babel}
\usepackage[autostyle,german=quotes]{csquotes}
\usepackage{setspace} % Package für Zeilenabstand

\onehalfspacing       %% 1,5-zeiliger Zeilenabstand
\parindent 0pt

\usepackage[
			left=30mm,
			right=30mm,
			top=30mm,
			bottom=30mm]{geometry}  % Einstellungen für die Seitenränder (links, rechts, oben, unten)

% This is a preliminary title for your thesis.
\newcommand{\thesisTitle}{Evaluierung von Systemen zur Speicherung und Bereitstellung von Binärdaten im Kontext von Web Services}

% Your full name.
\newcommand{\name}{Gamze Isik}

% The time frame in which you want to write the thesis.
\newcommand{\timeFrame}{20. März - \textbf{19. Juni 2023}}

% The primary supervisor. You probably do not need to change this.
\newcommand{\supervisor}{Erstbetreuer: Prof. Martin Goik}

% The name(s) of your advisor(s). This is probably the PhD student or PostDoc that you are working with.
\newcommand{\advisor}{Zweitbetreuer: Thomas Maier}

% Select if this is a Bachelor or Master Thesis.
\newcommand{\thesisType}{Bachelor}


%******************************************************************************	
% Ebenen Einstellungen
%1.section, 2.subsection, 3.subsubsection, 4.paragraph, 5.subparagraph
%******************************************************************************
\setcounter{secnumdepth}{4} % Anzahl der Ebenen für Überschriften einstellen
\setcounter{tocdepth}{3} % Anzahl der Ebenen für Überschriften im Inhaltsverzeichnis einstellen


%*******************************************************************************
% Grafiken
%******************************************************************************
\geometry{verbose}
\usepackage{graphicx}
\usepackage{float} 				% Zum Ausrichten von Tabellen und Grafiken
\usepackage[export]{adjustbox}

%****************************************************************************
% Tabellen
%****************************************************************************
\usepackage{tabularx}	% Ermöglicht Tabellen mit Autoumbruch

%%%%%%%%%%%%%%%%%%%%%%%%%%%%%%%%%%%%%%%%%%%%%%%%%%%%%%%%%%%%%%%%%%
% Helfer während der Texterstellung
%%%%%%%%%%%%%%%%%%%%%%%%%%%%%%%%%%%%%%%%%%%%%%%%%%%%%%%%%%%%%%%%%%
\usepackage{blindtext}      % Blindtext zum Testen von Textausgaben
\usepackage[]{showkeys} % Anzeige der Referenz-/Labelnamen im Text, Dokument auf final deaktiviert Option
\usepackage{fixme}  % ermöglicht die Verwendung von FixMes
\usepackage[pagewise]{lineno} % Zeilennummern beim Korrekturlesen sinnvoll

\usepackage[colorinlistoftodos,obeyDraft,backgroundcolor=red!40,bordercolor=red,linecolor=red,prependcaption]{todonotes}
% überschreibt die Voreinstellung "noinline" bei TODO Notes sodass diese immer Inline angezeigt werden
\makeatletter
\presetkeys{todonotes}{inline}{}
\makeatother


%%%%%%%%%%%%%%%%%%%%%%%%%%%%%%%%%%%%%%%%%%%%%%%%%%%%%%%%%%%%%%%%%%
%Kopf- und Fußzeile mit fancyhdr
%%%%%%%%%%%%%%%%%%%%%%%%%%%%%%%%%%%%%%%%%%%%%%%%%%%%%%%%%%%%%%%%%%
\usepackage{fancyhdr}
\fancyhf{}
%% Einstellungen für Kopf- und Fußzeile
\fancyhead[L]{\nouppercase{\leftmark}} 
\fancyfoot[C]{\thepage}

%%Linien für Kopf und Fußzeile sowie die Breite
\renewcommand{\headrulewidth}{0pt}
\usepackage[clearempty]{titlesec} % Leerseiten ohne Kopfzeile und ohne Nummerierung!

\titleformat{\chapter}[block]{\normalfont\huge\bfseries}{\thechapter}{10pt}{\huge}
\titlespacing{\chapter}{0pt}{0pt}{20pt}


%%%%%%%%%%%%%%%%%%%%%%%%%%%%%%%%%%%%%%%%%%%%%%%%%%%%%%%%%%%%%%%%%%
%Literaturverzeichnis
%%%%%%%%%%%%%%%%%%%%%%%%%%%%%%%%%%%%%%%%%%%%%%%%%%%%%%%%%%%%%%%%%%
\usepackage[
	style= authortitle-ibid, 	%Zitierstil
	bibstyle=numeric, 		%Biblografiestil
	backend=biber,
	sortlocale=de_DE,
	bibencoding=utf8,
	hyperref,
	defernumbers=true,
	ibidtracker=context, 	% damit ebd. funktioniert
	isbn=false, 			% ISBN im Literaturverzeichnis
	url=true, 				% Url im Literaturverzeichnis
	doi=false]{biblatex} 

% Lädt Anpassungen für den Zitierstil/Biblografiestil
% Hier sind Anpassungen des Zitierstil authortitle-ibid,  und des Biblografiestil numeric an die Verwendung von Fu�noten f�r Zitate.


% , als Trenner statt ; wenn Multicite z. B. footcites angewendet wird
%\renewcommand*{\multicitedelim}{\addcomma\space}
    
%Eigener Zitierstil     
% Author: Title (Year), [Nr.], S.X     
%\makeatletter
\renewbibmacro*{cite:title}{%
  %\cbx@tempa
  %\printtext[bibhyperref]%
      \iffieldundef{year}
        {}%
         {\printtext\space\mkbibparens{\printfield{year}}%
      		}%
      		\printtext{\addcomma\space}%
          \printtext[bibhyperref]{% Falls der Title auch verlinkt werden soll hier auskommentieren
          	\mkbibbrackets{\printfield{labelnumber}}%
          } %
          } %
%\makeatother       

% Doppelpunkt zwischen Author und Titel in der Fu�note
\renewcommand*{\nametitledelim}{\addcolon\space}

% Doppelpunkt zwischen Author und Titel im Literaturverzeichnis
\renewcommand*{\labelnamepunct}{\addcolon\space}

% Titel im Literaturverzeichnis nicht mehr kursiv darstellen
\DeclareFieldFormat{title}{#1\isdot}
\DeclareFieldFormat{citetitle}{#1\isdot}

\DeclareFieldFormat[article]{title}{#1\isdot}
\DeclareFieldFormat[article]{citetitle}{#1\isdot}
\DeclareFieldFormat{journaltitle}{#1\isdot}

\DeclareFieldFormat[thesis]{title}{#1\midsentence} 
\DeclareFieldFormat[thesis]{citetitle}{#1\midsentence}

% Literaturverzeichnis Anpassung

\DeclareBibliographyDriver{thesis}{%
  \usebibmacro{bibindex}%
  \usebibmacro{begentry}%
  \usebibmacro{author}%
  \setunit{\labelnamepunct}\newblock
  \usebibmacro{title}%
  \newunit
  \printlist{language}%
  \newunit\newblock
  \usebibmacro{byauthor}%
  \newunit\newblock
  \printfield{note}%
  \newunit\newblock
  \printfield{type}%
  \newunit
  \usebibmacro{publisher+location+date}%
  \newunit\newblock
  \usebibmacro{chapter+pages}%
  \newunit
  \printfield{pagetotal}%
  \newunit\newblock
  \iftoggle{bbx:isbn}
    {\printfield{isbn}}
    {}%
  \newunit\newblock
  \usebibmacro{doi+eprint+url}%
  \newunit\newblock
  \usebibmacro{addendum+pubstate}%
  \setunit{\bibpagerefpunct}\newblock
  \usebibmacro{pageref}%
  \usebibmacro{finentry}}

\DeclareBibliographyDriver{article}{%
  \usebibmacro{bibindex}%
  \usebibmacro{begentry}%
  \usebibmacro{author/translator+others}%
  \setunit{\labelnamepunct}\newblock
  \usebibmacro{title}%
  \newunit
  \printlist{language}%
  \newunit\newblock
  \usebibmacro{byauthor}%
  \newunit\newblock
  \usebibmacro{bytranslator+others}%
  \newunit\newblock
  \printfield{version}%
  \newunit\newblock
  \usebibmacro{in:}%
  \usebibmacro{journal+issuetitle}%
  \newunit
  \usebibmacro{byeditor+others}%
  \newunit
  \usebibmacro{note+pages}%
  \newunit\newblock
  \usebibmacro{publisher+location+date}% Zusatz um Verlag und Jahr anzugeben
  \newunit\newblock
  \iftoggle{bbx:isbn}
    {\printfield{issn}}
    {}%
  \newunit\newblock
  \usebibmacro{doi+eprint+url}%
  \newunit\newblock
  \usebibmacro{addendum+pubstate}%
  \setunit{\bibpagerefpunct}\newblock
  \usebibmacro{pageref}%
  \usebibmacro{finentry}}

\renewbibmacro*{journal+issuetitle}{%
  \usebibmacro{journal}%
  \setunit*{\addspace}%
  \iffieldundef{series}
    {}
    {\newunit
     \printfield{series}%
     \setunit{\addspace}}%
  %\printtext{Vol.\addspace}%
  \usebibmacro{volume+number+eid}%
   %\printtext{Nr.\addspace}%
   \printfield{issue}%
  %\setunit{\addspace}%
  \printtext{\addspace}%
  %\mkbibparens{\usebibmacro{date}}%
  %\usebibmacro{issue+date}%
  \setunit{\addcolon\space}%
  \usebibmacro{issue}%
  \newunit}

\DeclareBibliographyDriver{misc}{%
  \usebibmacro{bibindex}%
  \usebibmacro{begentry}%
  \usebibmacro{author/editor+others/translator+others}%
  \setunit{\labelnamepunct}\newblock
  \usebibmacro{title}%
  \newunit
  \printlist{language}%
  \newunit\newblock
  \usebibmacro{byauthor}%
  \newunit\newblock
  \usebibmacro{byeditor+others}%
  \newunit\newblock
  \printfield{howpublished}%
  \newunit\newblock
  \printfield{type}%
  \newunit
  \printfield{version}%
  \newunit
  %\printfield{note}%
  \newunit\newblock
  \usebibmacro{organization+location+date}%
    \newunit
  \printfield{note}%
  \newunit\newblock
  \usebibmacro{doi+eprint+url}%
  \newunit\newblock
  \usebibmacro{addendum+pubstate}%
  \setunit{\bibpagerefpunct}\newblock
  \usebibmacro{pageref}%
  \usebibmacro{finentry}}

\renewbibmacro*{organization+location+date}{
  \iffieldundef{year}
  {}%
  {
  \printtext{(}%
  \printfield{year}%
  \printtext{)}%
  }
  }
    %\printlist{location}%
  %\iflistundef{organization}
  %  {\setunit*{\addcomma\space}}
  %  {\setunit*{\addcolon\space}}%
  %\printlist{organization}%
  %\setunit*{\addcomma\space}%
  %\mkbibparens{\
    %\mkbibparens{\usebibmacro{date}}%
  %\newunit}

\renewbibmacro*{publisher+location+date}{%
  \printlist{location}%
  \iflistundef{publisher}
    {\setunit*{\addcomma\space}}
    {\setunit*{\addcomma\space}}%
  \printlist{publisher}%
  \printtext{\space}%
  %\setunit*{\addcomma\space}%
  \mkbibparens{\usebibmacro{date}}%
  \newunit}

% Aufteilung des Quellenverzeichnis in verschiedene Abschnitte. 
\defbibheading{literatur}{\subsection*{Literaturquellen}}
\defbibheading{pdf}{\subsection*{Elektronische Dokumente}}
\defbibheading{online}{\subsection*{Internetquellen}}

\addbibresource{./Bibtex/Quellen.bib}

%%%%%%%%%%%%%%%%%%%%%%%%%%%%%%%%%%%%%%%%%%%%%%%%%%%%%%%%%%%%%%%%%%
%Stichwortverzeichnis
%%%%%%%%%%%%%%%%%%%%%%%%%%%%%%%%%%%%%%%%%%%%%%%%%%%%%%%%%%%%%%%%%%%
\usepackage{idxlayout} %Damit der Abstand der Überschrift genauso hoch wie bei den Sonstigen Überschriften
\usepackage{makeidx}
\makeindex


% Verlinkung im Dokument 
%%%%%%%%%%%%%%%%%%%%%%%%%%%%%%%%%%%%%%%%%%%%%%%%%%%%%%%%%%%%%%%%%%
% Einstellungen für das erstellte PDF
\usepackage[
	bookmarks=true,
	bookmarksopen=true,
	bookmarksnumbered=true,
	pdftitle={Evaluierung von Systemen zur Speicherung und Bereitstellung von Binaerdaten im Kontext von Web Services}, 
	pdfauthor={\name},
	pdfsubject={\thesisType Thesis: Evaluierung von Systemen zur Speicherung und Bereitstellung von Binaerdaten im Kontext von Web Services},
	pdfkeywords={Speichersysteme},
	breaklinks=true,
	colorlinks=true,
	linkcolor=black,			% links in blau, black
	anchorcolor=blue,			% black
	citecolor=blue, 			% black
	filecolor=blue,				% black
	menucolor=blue,				% black
	urlcolor=blue,				% black
	pdfpagelabels=true,
	pdfstartview=Fit,
	hypertexnames=true,
	draft=false,
	plainpages=false,
	pdfpagelabels,
	hyperfootnotes=false,
	breaklinks=true]{hyperref}

\usepackage[bottom,hang,multiple]{footmisc} % Fußnoten Einstellungen
\setlength{\footnotemargin}{0pt}
\newcommand\fnsep{\textsuperscript{,}} %Komma zum manuellen trennen von Footnotes

\usepackage[all]{hypcap} % Workaaround damit Links auf Abbildungen und Tabellen auf den Beginn und nicht das Ende der Abb/Tab zeigt

% damit Urls an jedem Buchstaben umgebrochen werden, sinnvoll vor allem im Literaturverzeichnis
\usepackage[]{url}
\def\UrlBreaks{\do\a\do\b\do\c\do\d\do\e\do\f\do\g\do\h\do\i\do\j\do\k\do\l
\do\m\do\n\do\o\do\p\do\q\do\r\do\s\do\t\do\u\do\v\do\w\do\x\do\y\do\z\do\0
\do\1\do\2\do\3\do\4\do\5\do\6\do\7\do\8\do\9\do\-\do\_\do\I}%
\urlstyle{same}

%%%%%%%%%%%%%%%%%%%%%%%%%%%%%%%%%%%%%%%%%%%%%%%%%%%%%%%%%%%%%%%%%%
%Abbildgungen, Tabellen & Formeln pro Kapitel durchnummerieren 
%Tabelle 1 -> Tabelle 2.1
%%%%%%%%%%%%%%%%%%%%%%%%%%%%%%%%%%%%%%%%%%%%%%%%%%%%%%%%%%%%%%%%%%
\makeatletter
\@addtoreset{figure}{section}
\@addtoreset{table}{section}
\makeatother
\renewcommand{\thefigure}{\thesection.\arabic{figure}} % Abbildung
\renewcommand{\thetable}{\thesection.\arabic{table}}  % Tabelle
\renewcommand{\theequation}{\arabic{section}.\arabic{equation}} %Formeln


%%%%%%%%%%%%%%%%%%%%%%%%%%%%%%%%%%%%%%%%%%%%%%%%%%%%%%%%%%%%%%%%%%
%	Glossaries: Abkürzungsverzeichnis
%%%%%%%%%%%%%%%%%%%%%%%%%%%%%%%%%%%%%%%%%%%%%%%%%%%%%%%%%%%%%%%%%%
\usepackage[
   acronym, % Abkürzungsverzeichnis erstellen
   toc, % zum Inhaltsverzeichnis hinzufügen
   shortcuts]{glossaries} 

% Name des Abkürzungsverzeichnis anpassen
\addto\captionsngerman{\renewcommand\glossaryname{Glossar}}
\deftranslation[to=ngerman]{Acronyms}{Abkürzungsverzeichnis}
\deftranslation[to=ngerman]{Glossary}{Glossar}

%-----------------------------------------------------------------
% Eigener Stil für Abkürzungsverzeichnis
%-----------------------------------------------------------------
\newglossarystyle{mylist}{%
  \renewenvironment{theglossary}%
     {\begin{longtable}[l]{@{}lp{0.75\textwidth}@{}}} %Damit Tabelle Linksbündig am Rand beginnt
     {\end{longtable}}%
  \renewcommand*{\glossaryheader}{}%
  \renewcommand*{\glsgroupheading}[1]{}%
  \renewcommand*{\glossaryentryfield}[5]{%
    \glsentryitem{##1}\textbf{\glstarget{##1}{##2}} & ##3\glspostdescription\space ##5\\}%
  \renewcommand*{\glossarysubentryfield}[6]{%
     &
     \glssubentryitem{##2}%
     \glstarget{##2}{\strut}##4\glspostdescription\space ##6\\}%
%  \renewcommand*{\glsgroupskip}{ & \\}%
	 \renewcommand{\glsgroupskip}{}%
}
%-----------------------------------------------------------------
\renewcommand*{\glspostdescription}{} %Punkt am Ende jeder Beschreibung deaktivieren
\makeglossaries % Glossar erstellen



%************************************************************************************************
\begin{document}

%\linenumbers 			%Zeilennummern; zum Korrekturlesen einkommentieren

\pagestyle{empty}		% Keine Kopf-/Fusszeilen auf den ersten Seiten.

%****************************************
%Include
%****************************************
% Deckblatt
\begin{titlepage}
\begin{center}

\begin{minipage}[b]{.25\linewidth}
    \centering
    \includegraphics[width=\linewidth]{./Pictures/HdM_Logo.jpg}
\end{minipage}

\normalsize{Fakultät Druck und Medien}\\
\large{\textbf{Studiengang Medieninformatik}}\\[0.5cm]

\LARGE{{\thesisType} Thesis}\\
\normalsize{zur Erlangung des akademischen Grades Bachelor of Science}\\[0.7cm]
\Huge{\textbf{\thesisTitle}}

%\vspace{0.5cm}

%\Large{in Zusammenarbeit mit Leomedia GmbH}

\vspace{1cm} 

\Large{\textbf{\name}} \\[3pt]  

\large{\textbf{19. Juni 2023}}

\vspace{0.3cm} 

\large{Matrikelnummer: 39307} \\
\large{Bearbeitungszeitraum: \timeFrame} \\ 

\vspace{1cm}

\large{\textbf{Betreuer}}\\
\vspace{0.2cm}
\supervisor\\
\normalsize{Hochschule der Medien}\\
\vspace{0.2cm}
\large{\advisor}\\
\normalsize{Leomedia GmbH}

%\includegraphics[width=0.13\linewidth, right]{./Pictures/Löwe.png}
%\includegraphics[width=0.25\linewidth, right]{./Pictures/Leomedia_Logo.png}

\end{center}
\end{titlepage}
\vfill		% Deckblatt
% Erklärung

\section*{Erklärung}

Hiermit erkläre ich, dass ich die vorliegende Arbeit selbständig angefertigt habe. Es wurden nur die in der Arbeit ausdrücklich benannten Quellen und Hilfsmittel benutzt. Wörtlich oder sinngemäß übernommenes Gedankengut habe ich (mit Ausnahme dieser Erklärung) als solches kenntlich gemacht.
\vspace{4\baselineskip}

\begin{tabular}{lp{2em}l}
 \hspace{5cm}   && \hspace{4cm} \\
 \cline{1-1}\cline{3-3}
 Ort, Datum     && Unterschrift
\end{tabular} 
\thispagestyle{empty} 	% Erklärung
% Kurzfassung, Abstract
\renewcommand\abstractname{Zusammenfassung}

\begin{abstract}
Das Ziel der vorliegenden Arbeit ist die Evaluierung eines geeigneten Systems zur Speicherung und Bereitstellung von Binärdaten im Kontext von Web Services. Dabei werden die Anforderungen wie Performance, Verfügbarkeit, Sicherheit und API Anbindung gestellt. Durch die Realisierung eines Prototyps anhand der ausgewählten Speicherlösung werden die Binärdaten durch sichere, zeitlich begrenzte URL's bereitgestellt. Folgende Fragen werden gestellt: Welches Speichersystem ist im Hinblick auf Kosten, Performance und Verfügbarkeit für die Persistenz von Binärdaten besonders geeignet? Wie kann man Daten durch sichere, zeitlich begrenzte URL’s bereitstellen? Verschiedene Speichersysteme werden verglichen und anhand von Kostenkalkulationen bewertet. Durch die Auswertung des Vergleichs wird der Prototyp implementiert und Messungen auf Testdaten durchgeführt, welches die wissenschaftliche Arbeit stützt.*Das Ergebnis kann dazu genutzt werden, auf neue Speicherlösungen mit höherer Performance und Sicherheit mit akzeptablen Kosten umzusteigen.
\end{abstract}

\renewcommand\abstractname{Abstract}      

\begin{abstract}
The aim of this thesis is to evaluate a suitable system for storing and providing binary data in the context of web services. Requirements such as performance, availability, security, and API integration are set. By implementing a prototype based on the selected storage solution, binary data is provided through secure, time-limited URLs. The following questions are addressed: Which storage system is particularly suitable for persisting binary data in terms of cost, performance, and availability? How can data be provided through secure, time-limited URLs? Different storage systems are compared and evaluated based on cost calculations. By evaluating the comparison, the prototype is implemented and measurements are taken on test data, which supports the scientific work. The result can be used to switch to new storage solutions with higher performance and security at acceptable costs.
\end{abstract}

\clearpage

 		% Abstract
\include{./Inhaltsverzeichnis/Inhaltsverzeichnis}

% Abkürzungsverzeichnis
\newacronym{dt.}{dt.}{deutsch}
\newacronym{engl}{engl.}{englisch}
\newacronym[description={United States of America, \acs{dt.} Vereinigte Staaten von Amerika}]{USA}{USA}{United States of America}

\newacronym{Abb}{Abb.}{Abbildung}
\newacronym{Anm}{Anm.}{Anmerkung}
\newacronym{html}{HTML}{Hypertext Markup Language}
\newacronym{URL}{URL}{Uniform Resource Locator}
\newacronym{AWS}{AWS}{Amazon Web Services}
\newacronym{GCP}{GCP}{Google Cloud Platform}
\newacronym{S3}{S3}{Simple Storage Service}
\newacronym{GC}{GC}{Google Cloud}
\newacronym{GCS}{GCS}{Google Cloud Storage}
\newacronym{IAM}{IAM}{Identity and Access Management}
\newacronym{SSE}{SSE}{Server-Side Encryption}
\newacronym{SSE-S3}{SSE-S3}{Server-Side Encryption with S3 Managed Keys}
\newacronym{SSE-KMS}{SSE-KMS}{Server-Side Encryption with AWS Key Management Service}
\newacronym{SSE-C}{SSE-C}{Server-Side Encryption with Customer-Provided Keys}
\newacronym{API}{API}{Application Programming Interface}
\newacronym{SDK}{SDK}{Software Development Kit}
\newacronym{REST}{REST}{Representational State Transfer}
\newacronym{HTTP}{HTTP}{Hypertext Transfer Protocol}
\newacronym{CLI}{CLI}{Command Line Interface}
\newacronym{CRUD}{CRUD}{Create, Read, Update, Delete}
\newacronym{JSON}{JSON}{Javascript Object Notation}
\newacronym{XML}{XML}{Extensible Markup Language}
\newacronym{FUSE}{FUSE}{Filesystem in Userspace}
\newacronym{kb}{KB}{Kilobytes}
\newacronym{GB}{GB}{Gigabytes}
\newacronym{TB}{TB}{Terabytes}
\newacronym{CRR}{CRR}{Cross Region Replication}
\newacronym{SRR}{SRR}{Same Region Replication}
\newacronym{ADC}{ADC}{Application Default Credentials}	
\glsaddall[types={\acronymtype}] % damit auch nicht benutzte Abkürzungen erscheinen
\begin{singlespacing}	% Zeilenabstand 1.0
\printglossary[type=\acronymtype,style=mylist,nonumberlist]
\end{singlespacing}


%%%%%%%%%%%%%%%%%%%%%%%%%%%%%%%%%%%%%%%%%%%%%%%%%%%%%%%%%%%%%%%%%%
% Abbildungsverzeichnis ausgeben
%%%%%%%%%%%%%%%%%%%%%%%%%%%%%%%%%%%%%%%%%%%%%%%%%%%%%%%%%%%%%%%%%%
\clearpage
\phantomsection				
\addcontentsline{toc}{chapter}{\listfigurename}		
\listoffigures 

%%%%%%%%%%%%%%%%%%%%%%%%%%%%%%%%%%%%%%%%%%%%%%%%%%%%%%%%%%%%%%%%%%
% Tabellenverzeichnis ausgeben
%%%%%%%%%%%%%%%%%%%%%%%%%%%%%%%%%%%%%%%%%%%%%%%%%%%%%%%%%%%%%%%%%%
\phantomsection
\clearpage
\phantomsection		
\addcontentsline{toc}{chapter}{\listtablename}
\listoftables 
\clearpage

\chapter{Einleitung}

Das folgende Kapitel dient der Einführung in die Problemstellung, Motivation sowie Ziele und Vorgehensweisen der vorliegenden Arbeit.

\section{Problemstellung und Motivation}

Die steigende Menge an Daten im Kontext von Web Services, die in verschiedenen Anwendungen generiert werden, stellt eine große Herausforderung dar. Diese Bachelorarbeit richtet sich auf die Herausforderung einer Full-Service-Ticketing Software „leoticket“, die vom Unternehmen Leomedia GmbH entwickelt wird. 
Leomedia GmbH ist ein Unternehmen, das Software für Medienunternehmen wie Zeitungsverlage, Radiosender, Veranstalter, Künstler und Kulturvereine entwickelt.\footcite{leomedia-web} 
Leoticket ist eines der vielen Produkte von Leomedia, dass Services wie Online-Kartenvorverkäufe, Abendkassen, den Einlass bei der Veranstaltung, Statistiken, Abrechnungen und die Planung der Veranstaltung realisiert.\footcite{leomedia-web}\\ 

Die erfassten Daten umfassen verschiedene Arten von Dokumenten, wie beispielsweise PDF-Dateien, insbesondere Tickets und Rechnungen, die aus dem Kaufprozess über die leoticket Platform resultieren. Dabei ist es von großer Bedeutung, dass diese Daten von leoticket-Kunden und Leomedia sicher, zuverlässig und schnell gespeichert und abgerufen werden können. Vor diesem Hintergrund stellen sich Fragen nach der Auswahl eines geeigneten Speichersystems, welches die Performance, Verfügbarkeit, Sicherheitsanforderungen und die Möglichkeit der Integration in Software-Produkten wie leoticket erfüllt. Zudem müssen Mechanismen bereitgestellt werden, um den Zugriff der Ticketbesitzer auf die Daten durch sichere, zeitlich begrenzte URL’s zu beschränken.\\

Die Herausforderung von leoticket betrifft die Speicherung und Bereitstellung von Daten wie Tickets und Rechnungen an die Ticketkäufer. Ein Galera Cluster ist eine Multi-Master-Replikationslösung für relationale Datenbanken. Es basiert auf dem Konzept der synchronen Replikation, bei dem mehrere Knoten zu einem Cluster miteinander verbunden. 
Im Rahmen des Replikationsprozesses werden Daten synchronisiert, wodurch sich die Leistung bei Anfragen verringert, da das Datenbanksystem durch die Ausführung des Replikations- bzw. Synchronisierungsjobs beansprucht wird. Dabei erreicht der Arbeitsspeicher seine Kapazitätsgrenze von 200 GB. Die Hauptaufgaben der Datenbank umfassen beispielsweise die Durchführung von JOINS.\\ 

Eine weitere Herausforderung ist die Bereitstellung der Dateien über Email Anhänge. Anhänge dürfen eine bestimmte Speichergröße nicht überschreiten. Wenn Ticketkäufer mehrere Tickets in einer Bestellung tätigen, dann müssen diese über Email Anhänge bereitgestellt werden.\\

Leomedia plant eine Neugestaltung der leoticket-Anwendung, bei der sie sich von der Galera Cluster Technologie lösen möchten. In dieser Untersuchung werden keine relationalen Datenbanken berücksichtigt. Relationale Datenbanksysteme basieren auf einem festen Schema, das vorab definiert werden muss. Dies kann problematisch sein, wenn die Struktur der erfassten Daten des Produkts leoticket häufig geändert oder erweitert werden muss. NoSQL-Datenbanken bieten in dieser Hinsicht oft mehr Flexibilität, da sie schemalos oder mit flexiblen Schemas arbeiten. Weitere Gründe sind die Skalierbarkeit, Komplexität und der Speicherplatzbedarf. Es ist wichtig anzumerken, dass diese Gründe nicht bedeuten, dass relationale Datenbanksysteme grundsätzlich ausgeschlossen werden sollten. Sie sind nach wie vor eine bewährte und weit verbreitete Lösung für viele Anwendungen. Die Entscheidung, ein relationales Datenbanksystem auszuschließen, hängt von den spezifischen Anforderungen und Herausforderungen ab, die vorab für diese Arbeit definiert wurden.

\newpage

\section{Zieldefinition und Vorgehensweise}

Das Ziel dieser Arbeit besteht darin, anhand der Anforderungen von leoticket eine geeignete Speicherlösung zu empfehlen und die Realisierung eines Prototypen basierend auf den ausgewählten Cloud-Providern zu erstellen. Der Prototyp soll die Bereitstellung vordefinierter Daten durch sichere und zeitlich begrenzte URLs ermöglichen. 

Dabei werden folgende Fragen gestellt:

\begin{itemize}
	\item Welches Speichersystem ist im Hinblick auf Kosten, Performance und Verfügbarkeit für die Persistenz von Daten besonders geeignet? 
	\item Wie können diese Daten durch sichere, zeitlich begrenzte URL's bereitgestellt werden?
\end{itemize}

Zur Beantwortung werden verschiedene Arten von Speichersystemen untersucht. Dabei erfolgt eine Analyse  aktuell verfügbarer Speichertechnologien hinsichtlich ihrer nicht-funktionalen Eigenschaften wie Sicherheit, Verfügbarkeit, Performance und Kosten, die im Zusammenhang mit den verwendeten Technologien entstehen können. Aus funktionaler Sicht wird auch die Möglichkeit der Integration des Speichersystems in Software Produkten und die sichere Bereitstellung der erfassten Dateien betrachtet. Cloud-Provider werden miteinander verglichen und Kosten-, und Performance-Kalkulationen durchgeführt. Die Ergebnisse werden anschließend ausgewertet und interpretiert, um eine geeignete Speicherlösung zu empfehlen.\\

Der Prototyp wird auf Basis der von Cloud Providern angebotenen ausgewählten Technologien und Empfehlung implementiert. Nach der Durchführung von Messungen zur Performance auf Testdaten erfolgt eine Zusammenfassung der Implementierung.\\

Zum Abschluss werden die Ergebnisse präsentiert, interpretiert und eine Bewertung des Prototyps vorgenommen.  Die am Anfang gestellten Fragen der Arbeit werden beantwortet und potenzielle Anwendungen des Prototyps aufgelistet.

  
\chapter{Speichersysteme}

Speichersysteme sind eine entscheidende Komponente der IT Infrastruktur eines Unternehmens. In der heutigen Zeit kann man von Speichersystemen kaum absehen, da Big Data immer an Wichtigkeit gewinnt. Sie bieten eine Möglichkeit, große Mengen an Daten zu speichern und zu verwalten, um den Zugriff und die Nutzung zu erleichtern. Es gibt eine Vielzahl an Speichersystemen, die für verschiedene Zwecke konzipiert sind. Durch die große Auswahl in der IT und die stetig anwachsende Innovation stellt sich die Frage, welche Speichersysteme sich für bestimmte Zwecke (Use Cases) eignen. Die Wahl des richtigen Speichersystems hängt von den Anforderungen des Unternehmens ab, wie zum Beispiel der Art der zu speichernden Daten, dem benötigten Zugriff und die Skalierbarkeit. Eine gründliche Analyse der Anforderungen und Kosten ist entscheidend, um die beste Lösung zu finden, die den Bedürfnissen des Unternehmens entspricht.\\
\\ 
Im nachfolgenden Kapitel werden die unterschiedlichen Typen von Speichersystemen vorgestellt, wobei der Fokus auf den drei Speicherarten File-, Object- und Block Storage liegt. Im Anschluss daran erfolgt ein Vergleich von zwei Cloud-Providern in Bezug auf sicherer Speicherung, Hochverfügbarkeit, Performance, Kosten, API-Anbindung sowie der Dateibereitstellung. Auf Basis dieser Kriterien wird eine Entscheidung darüber getroffen, welcher Provider den Bedürfnissen von "Leoticket" entspricht. Hierbei fließen die Kosten- und Performance-Analysen mit in die Entscheidung ein.

\section{Arten von Speichersystemen}

Im folgenden Abschnitt werden die verschiedenen Arten von Speichersystemen vorgestellt, die für die Speicherung von digitalen Daten verwendet werden. Hierbei werden die drei gängigsten Speicherarten File-, Object-, und Blob Storage behandelt.\\
 
Die heutige IT-Landschaft bietet eine Vielzahl von Speichersystemen, die je nach Bedarf und Anforderungen ausgewählt werden können. Neben den traditionellen Speichermedien wie Festplatten und Bandlaufwerken gibt es heute auch verschiedene Arten von Speichersystemen, die in der Cloud oder als lokale Lösungen bereitgestellt werden können. Dazu gehören unter anderem File Storage, Object Storage und Blob Storage.\\

Jeder dieser Speicherarten hat ihre spezifischen Vor- und Nachteile und ist für bestimmte Anwendungsfälle besser geeignet als andere. 

\newpage
 
\subsection{File Storage}

File Storage, auch dateiebenen- oder dateibasierter Storage genannt \cite{redHat-storage}, ist eine Speicherlösung bei der Dateien auf einem Dateisystem gespeichert werden. 

\begin{quote}
	Dieses System wird auch als hierarchischer Storage bezeichnet und gilt als das älteste und am weitesten verbreitete Datenspeichersystem für Direct und Network-Attached Storage. Dateisysteme organisieren Daten in hierarchischen Ordnern und Unterverzeichnissen, ähnlich wie in einem Dateiordner auf einem Computer. Dateien werden in der Regel auf einem Server oder einer Festplatte gespeichert und können von mehreren Benutzern gleichzeitig gelesen und geschrieben werden. Hierbei werden die Informationen in einzelnen Verzeichnisse abgelegt und können über den entsprechenden Pfad aufgerufen werden. Um dies zu ermöglichen, werden begrenzte Mengen an Metadaten genutzt, die dem System den genauen Standort der Dateien mitteilen, vgl. \citeauthor{redHat-storage}.
\end{quote}

In der folgenden Abbildung wird die hierarchische Struktur des Dateispeichers visualisiert.\\

\begin{figure}[h]
\centering
	\includegraphics{Pictures/FileStorageHierarchy.png}
	\caption{File Storage: Aufbau des Hierarchiesystems, \citeauthor{redHat-storage}}
\end{figure}

File Storage wird häufig in Unternehmen und Organisationen eingesetzt, um gemeinsame Dateiserver bereitzustellen oder Daten in Cloud-Speicherdiensten wie Dropbox oder Google Drive zu speichern.
Auch wenn es von Betriebssystemen und Anwendungen gut unterstützt wird, kann die Performance und Skalierbarkeit von File Storage bei sehr großen Dateisystemen beeinträchtigt werden, was insbesondere bei stark frequentierten Anwendungen oder bei der Verarbeitung großer Datenmengen zum Problem werden kann. 

\begin{quote}
	Mit zunehmenden Datenvolumen erfordert das Skalieren von Dateispeichern das Hinzufügen neuer Hardwaregeräte oder den Austausch vorhandener Geräte durch solche mit höherer Kapazität. Dies kann im Laufe der Zeit teuer werden. \glqq As data volumes expand, scaling file storage requires [...]\grqq, (\cite{nx-fileScala}, Übersetzung des Autors)
\end{quote}

Laut Wahlmann (\citeyear{nx-fileScala}, Übersetzung des Autors) wird die Datenspeicherung bei zu vielen Daten nicht nur teuer, sondern auch unhandlich und zeitaufwändig. Der schnelle und einfache Zugriff auf jede einzelne Datei wird schwierig, wenn man zig Millionen von Dateien in Tausenden von Verzeichnissen auf Hunderten von Speichergrößen speichert. 

\subsection{Block Storage}

Block Storage, auch genannt als Block-level Storage speichert Dateien auf SAN(Storage Area Networks) basierten Netzwerken oder auf Cloud-basierten Speicherumgebungen. Das System teilt Daten in Blöcke auf und speichert die separaten Teile jeweils mit einer eindeutigen Kennung, vgl. \cite{ibm-topics}.\\

Daten werden in Blöcken auf dem Datenträger gespeichert, die unabhängig voneinander adressiert werden können. Jeder Block ist eine feste Größe, typischerweise im Bereich von einigen Kilobytes bis hin zu einigen Megabytes. Diese Blöcke können im System an jeder Stelle gespeichert werden.

\begin{quote}
	Wenn auf Block Storage gespeicherte Daten abgerufen werden, verwendet das Server-Betriebssystem die eindeutige Adresse, um die Blöcke wieder zusammenzufügen und so die Datei zu erstellen. Der Vorteil besteht darin, dass das System nicht durch Verzeichnisse und Dateihierarchien navigieren muss, um auf die Datenblöcke zuzugreifen. Dadurch werden Effizienzen erzielt, da der Abruf von Daten schneller erfolgen kann, vgl. \cite{ibm-storage}.
\end{quote}

Typische Anwendungsbereiche des Block Storage sind Datenbanken, Virtualisierungsumgebungen und Anwendungen für Big Data-Analysen. Speicherung von strukturierten Daten wie Datenbanken, Virtuelle Maschinen und Betriebssysteme eignen sich besonders bei der Verwendung von Block Storage. Diese Art von Daten erfordert schnellen und direkten Zugriff auf bestimmte Bereiche des Speichers und muss häufig in Echtzeit ausgeführt werden. Block Storage eignet sich daher am besten für Anwendungen mit hohen Anforderungen an die Leistung und niedriger Latenzzeit.

\newpage

\subsection{Object Storage}

Object Storage hat sich als Speichertechnologie in den letzten Jahren immer stärker etabliert und wird von vielen Unternehmen als Alternative zu traditionellen Speicherlösungen wie Block- oder File Storage angesehen. Die ersten Object Storage Systeme wurden bereits in den 1990er Jahren entwickelt, aber erst mit dem Aufkommen von Big Data, IoT und der Cloud-Nutzung 
 hat es einen breiteren Einsatz gefunden. Heute bieten viele Cloud Provider wie Amazon Web Services (AWS) und Google Cloud Platform (GCP) Object Storage als einen ihrer Haupt-Cloud-Services an.\\
 
\begin{quote}
	Object Storage ist für den Umgang mit großen Datenvolumen und unstrukturierten Daten entwickelt wurden. Sie speichert Daten als eigenständige Objekte, die aus Daten und Metadaten bestehen und einen eindeutigen Identifier (UID) haben (\glqq Object storage (aka object-based storage) is a type of data storage used to [...]\grqq, \cite{dataCore-OS}, Übersetzung des Autors).
\end{quote}

Im Gegensatz zu hierarchischen Systemen wie beim File Storage ist das Speichersystem flach strukturiert. Durch die einfache API Anbindung kann es mit vorhandenen Anwendungen integriert werden. Nutzer können detaillierte Informationen wie beispielsweise Erstellerangaben, Schlüsselwörter sowie Sicherheit-und Datenschutzrichtlinien hinterlegen. Diese Daten bezeichnet man als Metadaten.\\

Laut \citeauthor{nx-fileScala}, 2022 ist Skalierbarkeit der Hauptvorteil, da bei der Speicherung von Petabyte und Exabyte alle Objekte in einem Namespace abgelegt werden. Selbst wenn dieser Namespace auf Hunderte von Hardwaregeräten und Standorten verteilt ist, können alle Objekte schnell abgerufen werden. Andere Vorteile von Objekt Storage beinhalten die Datenintegritätsprüfung, im Englischen bekannt als \glqq erasure coding\grqq und die Datenanalyse.\\

Auch Object Storage hat seine Nachteile. Laut \citeauthor{redHat-storage} muss das Objekt nach der Speicherung bei Veränderung komplett neu überschrieben werden. Sie sind für traditionelle Datenbanken nicht geeignet, da das Schreiben von Objekten Zeit beansprucht und man sich mit der API auseinandersetzen muss, vgl. \citeauthor{redHat-storage}.\\

Insgesamt bietet Object Storage eine skalierbare und flexible Methode zur Speicherung von unstrukturierten Daten. Organisationen sollten jedoch die Vor- und Nachteile von Object Storage im Kontext ihrer spezifischen Anwendungsfälle abwägen, um eine fundierte Entscheidung über die beste Speichermethode zu treffen.\\

%TODO Object Storage Entscheidung

Da leoticket Daten wie Rechnungen und Tickets als Dateien speichern und abrufen abruft, ist Object Storage die richtige Speicherart. 

Für leoticket ist die Object Storage Variante die am Besten geeignetste Speicherlösung, da ... %TODO Satz weiterschreiben


\section{Aktuelle Speichertechnologien im Markt}

Die beiden größten Cloud-Speicheranbieter, Amazon Web Services (AWS) und Google Cloud Platform (GCP), bieten eine Vielzahl von Speicherlösungen an, die auf die Bedürfnisse von Unternehmen zugeschnitten sind.\\ 

In diesem Kapitel werden die aktuellen Speichertechnologien auf dem Markt untersucht, wobei der Fokus auf den Angeboten von AWS und GCP liegt. Um eine Vergleichsgrundlage zwischen AWS und GCP zu schaffen, werden die verschiedenen Aspekte wie sichere Speicherung, Hochverfügbarkeit, Leistung, Kosten, API Anbindung und die Bereitstellung der Dateien betrachtet. Dieses Vorgehen dient dazu, zu ermitteln, welche Speicherart sich am besten für welche Anforderungen eignet. Als Kontrast dazu wird das Open-Source-Objektspeichersystem MinIO betrachtet.\\

AWS bietet eine Reihe von Speicheroptionen an, darunter Amazon S3 (Simple Storage Service). Amazon S3 ist ein Object Storage-Service, der für die Speicherung und den schnellen Abruf von unstrukturierten Daten wie Videos, Fotos und Dokumenten ausgelegt ist.

Google Cloud Platform bietet ebenfalls eine Vielzahl von Speicherlösungen an, darunter Google Cloud Storage. Google Cloud Storage ist ein Object Storage-Service, der für die Speicherung von unstrukturierten Daten wie Bildern, Videos und Dokumenten ausgelegt ist.\\
 
Insgesamt bieten AWS und Google Cloud Platform eine Vielzahl von Speicherlösungen an, die auf die Bedürfnisse von Unternehmen zugeschnitten sind. Organisationen sollten jedoch die Vor- und Nachteile jeder Speicherlösung abwägen, um die beste Lösung für ihre spezifischen Anforderungen zu finden.

\newpage

\subsection{Eigenschaften}

Im vorliegenden Abschnitt werden Amazon S3 und Google Cloud Storage in Bezug auf verschiedene Kriterien untersucht. Dabei werden zunächst die Eigenschaften von AWS S3 erläutert, gefolgt von einer Betrachtung von GC Storage. Ziel ist es, die Unterschiede zwischen den beiden Anbietern hervorzuheben und eine Entscheidungshilfe zu bieten, welche Plattform für die Anforderungen von \glqq leoticket\grqq am besten geeignet ist. Die Auswahl der Kriterien erfolgt in Anlehnung an die spezifischen Anforderungen von \glqq leoticket\grqq.

%TODO Aws und GCP Definition?

\subsubsection{Sichere Speicherung}

\textbf{Amazon S3}\\

Viele Anbieter von Object Storage-Lösungen bieten integrierte Verschlüsselungsmöglichkeiten an, um sicherzustellen, dass Daten sowohl in Ruhe als auch in Bewegung geschützt sind. Dabei können unterschiedliche Verschlüsselungsmethoden zum Einsatz kommen. In Bezug auf die sichere Speicherung bieten sowohl AWS als auch GCP verschiedene Optionen für die Verschlüsselung von Daten.Die Sicherheit der gespeicherten Daten ist von entscheidender Bedeutung, um die Integrität und Vertraulichkeit der Daten zu gewährleisten. In diesem Unterabschnitt werden die verschiedenen Sicherheitsfunktionen von Amazon S3 untersucht.\\

IAM (Identity and Access Management) ist ein wichtiger Bestandteil von AWS und ermöglicht es Benutzern, Gruppen und Rollen zu erstellen, um den Zugriff auf S3 zu verwalten. Benutzer können individuelle Berechtigungen zugewiesen werden, während Gruppen und Rollen mehrere Benutzer mit denselben Berechtigungen zusammenfassen können. Auf S3-Buckets und Objekte kann man eine granuläre Zugriffssteuerung anwenden. Benutzer, Gruppen oder Rollen können so berechtigt werden, den Zugriff auf bestimmte Buckets und Objekte zu beschränken oder zu erlauben. "Beim Erteilen von Berechtigungen in Amazon S3 entscheiden Sie [...]."\cite{aws-iam-s3}\\

Eine weitere wichtige Sicherheitsfunktion von Amazon S3 ist die Datenverschlüsselung. S3 bietet eine Vielzahl von Verschlüsselungsoptionen für die serverseitige und clientseitige Verschlüsselung. Da für leoticket die clientseitige Verschlüsselung nicht in Frage kommt, liegt der Fokus auf der serverseitigen Verschlüsselung. Es gibt drei Methoden, darunter die serverseitige Verschlüsselung mit Amazon S3-verwalteten Schlüsseln (SSE-S3), mit KMS-verwalteten Schlüsseln (SSE-KMS) und die SSE-C. Diese Optionen ermöglichen es Benutzern, die Verschlüsselung auf ihre spezifischen Anforderungen abzustimmen und so die Sicherheit der Daten zu gewährleisten.

Laut \citeauthor{aws-iam-s3} nutzen Buckets standardmäßig die SSE-S3 Methode. Für die Verschlüsselung wird die 256-bit Advanced Encryption Standard (AES-256) verwendet. Seit dem 5. Januar 2023 sind alle neu erstellen Buckets auf SSE-S3 ausgelegt. Alle neuen Objekte sind beim Hochladen automatisch verschlüsselt, ohne weitere Zusatzkosten und keine Einbußung der Leistung.

\begin{quote}
	Die Serverseitige Verschlüsselung schützt die Daten at rest. Amazon S3 verschlüsselt jedes Objekt mit einem eindeutigen Schlüssel. Als zusätzliche Sicherheitsmaßnahme werden diese eindeutigen Schlüssel mit einem weiteren Schlüssel verschlüsselt, welches in regelmäßigen Abständen rotiert wird, vgl. \cite{aws-iam-s3}
\end{quote}

Amazon S3 stellt auch die SSE-KMS als Auswahl zur Verfügung. AWS-KMS ist ein Dienst, dass einen Schlüsselverwaltungssystem zur Verfügung stellt. Es verschlüsselt die Objekt Daten und speichert die S3 Checksum, das im Objekt Metadaten steckt, in verschlüsselter Form. Man kann die SSE-KMS in der AWS Management Konsole oder über die AWS KMS API verwalten. Jedoch gibt es zusätzliche Kosten bei der Verwendung der Methode. Dazu mehr im Kosten Abschnitt. Bei der Nutzung von AWS-SSE gibt es zwei Methoden. Einmal die AWS managed key oder die customer managed key. Es unterstützt die \glqq envelope encryption\grqq. Das bedeutet, das die Schlüssel für die Daten durch einen Master Key verschlüsselt wird. Dies erleichtert die Verwaltung der Schlüssel.\\

Bei der AWS managed key Variante wird automatisch ein Schlüssel erstellt, sobald ein Objekt in ein Bucket hochgeladen wird. Dieser generierte Schlüssel wird dann für die Ver- und Entschlüsselung der Daten verwendet. Wenn man einen eigenen Schlüssel bei KMS verwenden möchte, dann erstellt man zuerst einen symmetrischen Key vor der KMS Konfiguration. Bei der Bucket Erstellung kann man anschließend den selbst-erstellten Key angeben. Die Nutzung von Customer Managed Keys hat einige Vorteile, die den Anforderungen von leoticket entspricht. Selbsterstellte Schlüssel bieten mehr Flexibilität und Kontrolle. Man kann sie selber erstellen, rotieren und deaktivieren. Hinzufügend kann man auch Zugriffskontrollen und Auditierung für den Schutz der Daten konfigurieren.\\

Wenn man die SSE-KMS Variante aussucht, kann man auch die S3 Bucket Key Funktion aktivieren. Dies kann die Request Kosten bis zu 99 Prozent reduzieren, indem die Request Traffic von Amazon S3 zu AWS KMS reduziert wird. Durch die Aktivierung der S3 Bucket Key für einen Bucket werden unique data keys für die Objekte im Bucket erstellt. Diese Bucket Keys werden für eine bestimmte Zeit verwendet, welches Abrufe zu Amazon S3 nach AWS KMS reduziert um Verschlüsselungsoperationen durchzuführen.\\

Zuletzt gibt es noch die SSE-C (server-side encryption with customer-provided keys). Bei der SSE-C stellt der Customer seinen eigenen Schlüssel zur Verfügung. Dieser Schlüssel wird nicht von Amazon S3 gespeichert, im Gegensatz bei AWS KMS. Amazon S3 übernimmt mit dem bereitgestellten Schlüssel die Datenverschlüsselung beim Schreiben sowie die Datenentschlüsselung beim Zugriff auf Objekte. Amazon S3 entfernt anschließend den Schlüssel aus dem Speicher. Da Amazon S3 den Schlüssel nicht speichert, sepeichert er den zufällig generierten HMAC (Hash-baded Message Authentication Code) Wert vom Encryption Key um zukünftige Requests zu validieren."Note: Amazon S3 does not store the encryption key that you provide. [...]", \cite{aws-sse-c}.\\

\newpage

Object Ownership ist eine weitere wichtige Sicherheitsfunktion von Amazon S3. Mit Object Ownership können Benutzer oder Gruppen die Eigentümerschaft von Objekten in S3-Buckets besitzen. Dies bedeutet, dass nur autorisierte Benutzer die Berechtigung haben, Objekte zu löschen oder zu ändern, was die Sicherheit der Daten erhöht.\glqq S3 Object Ownership ist eine Einstellung auf Amazon-S3-Bucket-Ebene, mit der Sie Zugriffskontrolllisten (ACLs) deaktivieren und das Eigentum an jedem Objekt in Ihrem Bucket übernehmen können, [...]." \cite{aws-iam-s3}\\ 

AWS empfiehlt die ACL (Access Control List) auf Bucket-Ebenen deaktiviert zu lassen. Alle Objekte eines Buckets gehört so dem Bucket Owner. Laut \citeauthor{aws-iam-s3} verfügt Object Ownership über drei Einstellungen, mit denen man die Eigentümerschaft von Objekten steuern kann. 

\begin{figure}[h]
\centering
	\includegraphics[width=16cm,keepaspectratio]{Pictures/objectOwnershipTable.png}
	\caption{Einstellungen für Object Ownership, \citeurl{aws-iam-s3}}
\end{figure}

Laut \citeauthor{aws-iam-s3} zeigt die obige Tabelle die Auswirkungen, die jede Einstellung für Object Ownership auf ACLs, Objekte, Objekteigentümer und Objekt-Uploads hat.\\

Logging ist ein weiterer Aspekt der Amazon S3-Sicherheit. S3 bietet verschiedene Logging-Optionen, darunter Bucket Logging und Object-Level Logging, die es Benutzern ermöglichen, Zugriffe auf S3-Objekte aufzuzeichnen und zu überwachen. Diese Funktionen sind entscheidend, um Compliance-Anforderungen zu erfüllen und verdächtige Aktivitäten zu erkennen. AWS bietet eine Vielzahl von tools zur Überwachung der Amazon S3 Ressourcen. Darunter die Amazon CloudWatch Alarms, AWS CloudTrail Logs, Amazon S3 Access Logs und die AWS Trusted Advisor.\\

%TODO compliance

\newpage

\textbf{Google Cloud Storage}




\subsubsection{Hochverfügbarkeit}
\subsubsection{Performance}
\subsubsection{Kosten}
\subsubsection{API Anbindung}

\subsection{Bereitstellung der Dateien}

\section{Auswahl des Speichersystems}

In diesem Abschnitt wird durch eine Kosten-, und Performance Analyse die Speichersysteme ausgewertet und eine Auswahl getroffen. Die Kalkulationsergebnisse werden zum Schluss dargestellt.

\subsection{Kostenanalyse}
\subsection{Performance Analyse}
\subsection{Kalkulationsergebnisse}
\chapter{Prototypische Umsetzung}     

Im folgenden Kapitel wird der Prototyp genauer betrachtet und die verschiedenen Aspekte seiner Entwicklung und Implementierung werden erläutert. Es werden dabei die verwendeten Technologien und die Art der Speicherung und Bereitstellung von Binärdaten beleuchtet. Um die Performance zu bewerten, werden Messungen auf generierte Testdaten durchgeführt. Insgesamt dient das Kapitel als Grundlage für weitere Untersuchungen und die Optimierung des Prototyps.\\                               

\section{Überblick und Vorgehensweise}

Zunächst wird auf die eingesetzten Technologien eingegangen, die bei der Entwicklung des Prototyps verwendet werden. Dies umfasst das Framework Spring Boot und Programmiersprachen, die zur Umsetzung des Prototyps genutzt werden. Ein besonderes Augenmerk wird auf die Speicherung der Binärdaten gelegt. Hier werden verschiedene Ansätze betrachtet, wie beispielsweise die Verwendung von AWS SDK und Google Client Libraries. Des Weiteren wird die Bereitstellung der Binärdaten behandelt. Hier wird die Methode des Signed URLs betrachtet, um die Daten effizient an die Anwender zu übertragen. Um die Leistung des Prototyps zu bewerten, werden Testdaten mit zufälligem Inhalt generiert. Dies ermöglicht eine realistische Simulation der Tickets und Rechnungen in leoticket und erlaubt eine Bewertung der Performance der Cloud Provider. Die Messungen werden auf einem virtuellen Server durchgeführt, um eine präzisere Analyse zu gewährleisten.\\

\newpage

\section{Eingesetzte Technologien}

Für die Umsetzung des Prototyps werden die folgenden Technologien eingesetzt:

\begin{itemize}
	\item Spring Boot v3
	\item AWS SDK 2.0 Version
	\item GC Storage client library
	\item Java SDK 17 Temurin Version
	\item Maven v4.0.0
	\item AWS Toolkit
	\item aws cli
	\item gcloud cli
\end{itemize}

Für die Implementierung wurde die Entwicklungsumgebung IntelliJ IDEA Ultimate verwendet. IntelliJ bietet ein Plugin namens \verb|AWS Toolkit| an, das installiert werden kann. Als Framework wurde die aktuellste Version von Spring Boot (Version 3) zum Zeitpunkt der Erstellung des Prototyps verwendet. Spring Boot stellt SDKs beider Cloud-Anbieter als Maven-Abhängigkeiten zur Verfügung.

\begin{quote}
	Am 17. März 2021 wurde die neue Version des Spring Cloud AWS 2.3 veröffentlicht. Spring Cloud GCP und Spring Cloud AWS sind nicht mehr Teil des Spring Cloud Releases. Nicht Teil des Releases zu sein bedeutet auch, dass sie aus der Spring Cloud Organisation auf Github herausgenommen worden sind und dadurch neue Maven Package Namen haben. Das neue Package für Spring Cloud AWS heißt nun \glqq io.awspring.cloud\grqq, vgl. \cite{spring-cloud-announce}. 
\end{quote}

Die unten aufgeführten Maven Abhängigkeiten werden für AWS S3 und Cloud Storage verwendet:

\begin{lstlisting}[language=XML]
	<dependency>
        	<groupId>com.google.cloud</groupId>
        	<artifactId>spring-cloud-gcp-starter-storage</artifactId>
    </dependency>
\end{lstlisting}

\begin{lstlisting}[language=XML]
	dependency>
        	<groupId>io.awspring.cloud</groupId>
        	<artifactId>spring-cloud-aws-s3</artifactId>
        	<version>3.0.0</version>
    </dependency>
\end{lstlisting}

Als Spring Cloud GCP Version wird die 4.2.0 verwendet. Beide Spring Cloud Abhängigkeien werden von der Community auf Github verwaltet und aktualisiert. Für die Erstellung des Spring Boot Projekts wurde der Spring Initializier von Spring selbst verwendet unter \url{https://start.spring.io/}. Zudem wird die Java SDK 17 Temurin Version verwendet. Für Maven wird die 4.0 Version verwendet. Für die Authentifizierung wird die \verb|gloud| cli verwendet. Diese wird über die offizielle Dokumentation installiert. Siehe \url{https://cloud.google.com/sdk/docs/install?hl=de}. Das AWS Toolkit wird für die Authentifizierung mit AWS angewendet. Um sich mit GC zu verbinden wird eine Methode des ADC verwendet, welches im nächsten Abschnitt genauer erläutert wird.

\newpage

\section{Speicherung von Binärdaten}

Um Daten in S3 oder Cloud Storage speichern zu können, wurde der Prototyp so implementiert, dass der Nutzer sich zwischen S3 oder Cloud Storage entscheiden kann. Dies geschieht über die Klasse \verb|CloudStorageServiceFactory|. Hier kann der Nutzer über die Umgebungsvariable \verb|cloud_provider| den gewünschten Provider mit \verb|aws| oder \verb|google cloud| angeben. Dabei wird die Groß-, und Kleinschreibung nicht berücksichtigt. Die Umgebungsvariablen können im System durch \verb|export <EnvironmentVariable>=<value>| exportiert werden. Wenn eine IDE wie Intellij verwendet wird, kann dies unter den Run-Einstellungen als Umgebungsvariablen eingefügt werden. Nach der Eingabe des Cloud Providers wird das Programm die entsprechende Klasse aufrufen. Für AWS die Klasse \verb|AWSS3StorageService| und für GC die Klasse \verb|GoogleCloudStorageService|. Beide Klassen implementieren von dem Interface \verb|CloudStorageService|. Die \verb|CloudStorageService| definiert zwei Methoden und eine davon ist für die Speicherung der Daten zuständig. Siehe folgenden Code Snippet:

\begin{lstlisting}[language=Java]
void uploadObject(String bucketName, String key, String file, String encryptionKey) throws IOException;
\end{lstlisting}
	
Dieser Methode wird der Bucket Name, der Name des Objekts, der Pfad des Objekts und der Encryption Key übergeben. Der Encryption Key kann dabei der Schlüssel sein, der in AWS KMS oder GC KMS generiert wurde. Die Implementierung dieser Methode ist für beide Cloud Provider ähnlich gestaltet.

\begin{lstlisting}[language=Java, caption=Prototyp Code Snippet - Hochladen eines Objekts nach S3]
@Override
public void uploadObject(String bucketName, String key, String file, String encryptionKey, String storageClass) {
    try {

        PutObjectRequest putObjectRequest = PutObjectRequest.builder()
                .bucket(bucketName)
                .key(key)
                .serverSideEncryption(ServerSideEncryption.AWS_KMS)
                .ssekmsKeyId(encryptionKey)
                .storageClass(storageClass)
                .build();

        Path filePath = Paths.get(file);
        byte[] fileBytes = Files.readAllBytes(filePath);
        RequestBody requestBody = RequestBody.fromBytes(fileBytes);

        this.s3Client.putObject(putObjectRequest, requestBody);

        System.out.println("File " + file + " uploaded to bucket " + bucketName + " as " + key);

    } catch (S3Exception | IOException e) {
        System.out.println(e.getMessage());
    }
}
\end{lstlisting}

\newpage

Der vorliegende Code (3.1) beschreibt den Vorgang des Speicherns eines Objekts in AWS S3. Zunächst wird ein PUT-Request-Objekt erstellt, wobei Parameter wie der Bucket-Name, der Objektname als Key und die Authentifizierungsmethoden angegeben werden. Dabei wird die AWS KMS-Methode verwendet und der entsprechende Schlüssel bereitgestellt. Anschließend wird die Speicherklasse angegeben, in der das Objekt gespeichert werden soll. In diesem Beispiel wird die Standard-IA-Klasse verwendet. Danach wird der Pfad der angegebenen Datei gelesen, in ein Byte-Array umgewandelt und dem \verb|RequestBody| übergeben. Der RequestBody wird gemeinsam mit dem \verb|PutObjectRequest| an den \verb|S3Client| übergeben und mit der AWS \verb|.putObject()| Methode in S3 hochgeladen. AWS S3 verschlüsselt das Objekt mit dem angegebenen Verschlüsselungsschlüssel und lädt es in S3 hoch.\\

Der folgende Code Snippet zeigt die Methode der Klasse \verb|GoogleCloudStorageService|. Ähnlich wie bei der Methode für AWS S3 wird auch hier ein Objekt nach Cloud Storage hochgeladen:

\begin{lstlisting}[language=Java, caption=Prototyp Code Snippet - Hochladen eines Objekts nach Cloud Storage]
@Override
public void uploadObject(String bucketName, String key, String file, String encryptionKey, String storageClass) throws IOException {

    Map<String, String> kmsKeyName = new HashMap<>();
    kmsKeyName.put("kmsKeyName", encryptionKey);

    BlobId blobId = BlobId.of(bucketName, key);
    BlobInfo blobInfo = BlobInfo.newBuilder(blobId)
            .setMetadata(kmsKeyName)
            .build();

    Storage.BlobWriteOption precondition;
        
    if (this.storage.get(bucketName, key) == null) {
        precondition = Storage.BlobWriteOption.doesNotExist();
            
    } else {
        precondition =
                Storage.BlobWriteOption.generationMatch(
                        this.storage.get(bucketName, key).getGeneration());
    }
        
    this.storage.createFrom(blobInfo, Paths.get(file), precondition);

    System.out.println("File " + file + " uploaded to bucket " + bucketName + " as " + key);
}
\end{lstlisting}

Dabei werden ähnliche Parameter der Methode wie in AWS S3 übergeben. Um ein Objekt in ein Bucket hochladen zu können, wird eine Referenz zum Bucket erstellt. Dies geschieht durch die \verb|BlobId|, der den Bucket Namen und den Namen des Objekts beinhaltet. Anschließend wird diese blobId dem \verb|BlobInfo| Objekt übergeben und die Speicherklasse \verb|NEARLINE| definiert. Anschließend wird über die Metadaten die KMS Encryption Key gesetzt. Nach dem überprüft worden ist, ob das Objekt bereits im Bucket existiert oder nicht, wird das Objekt in der Zeile 23 hochgeladen.\\

Um das Programm auszuführen, wird die Hauptklasse \verb|HandsonAwsGcApplication| gestartet. Damit das Programm erfolgreich läuft, müssen die Umgebungsvariablen im System exportiert werden. Alle Umgebungsvariablen sind in der \verb|application.properties| hinterlegt. Diese werden beim Start des Programms gelesen und angewendet. Außerdem müssen die AWS und GC Credentials hinterlegt werden. Dies kann entweder über die AWS Toolkit Plugin gesteuert werden oder mit dem Befehl \verb|aws configure| in der Kommandozeile. Für GC Credentials kann mit dem Befehl: \\ \verb|export GOOGLE_APPLICATION_CREDENTIALS=<service-account-json-file>| der Service Account hinterlegt werden oder durch Ausführen des Befehls \verb|gcloud auth application-default login| in der Kommandozeile, was die Credentials lokal speichert und für ADC verwendet wird.

\newpage

\section{Bereitstellung der Binärdaten}

Bei der Bereitstellung von Binärdaten in leoticket geht es darum, den Kunden Dateien über Links zugänglich zu machen. Hierfür werden signierte URLs verwendet. Im folgenden Abschnitt werden die Implementierungen und Erläuterungen des entsprechenden Codes von beiden Cloud Providern vorgestellt.\\

Der untere Code Snippet (3.3) zeigt die Methode der Klasse \textbf{AWSS3StorageService}:

\begin{lstlisting}[language=Java, caption=Prototyp Code Snippet - Generierung eines signierten URLs durch AWS]
@Override
public void getPresignedUrl(String bucket, String key, Integer minutes, String encryptionKey) {
    try {

        GetObjectRequest getObjectRequest = GetObjectRequest.builder()
                .bucket(bucket)
                .key(key)
                .build();

        GetObjectPresignRequest getObjectPresignRequest = GetObjectPresignRequest.builder()
                .signatureDuration(Duration.ofMinutes(minutes))
                .getObjectRequest(getObjectRequest)
                .build();

        PresignedGetObjectRequest presignedGetObjectRequest = presigner.presignGetObject(getObjectPresignRequest);

        String url = presignedGetObjectRequest.url().toString();

        System.out.println("Presigned URL: " + url);

    } catch (S3Exception e) {
        System.out.println(e.getMessage());
    }
}
\end{lstlisting}

Die Methode erhält ähnlich wie beim Hochladen des Objekts die Parameter Bucket-Name und Objektname. Zudem wird eine Zeitdauer in Minuten übergeben, die festlegt, wie lange der generierte Link gültig sein soll. Für die Entschlüsselung der Daten muss der gleiche Encrption Key wie beim Hochladen verwendet werden. In Zeile 5 wird das \verb|GetObjectRequest| Objekt erstellt und mit dem Bucket-Namen und dem Objektnamen versehen. Anschließend wird dieses Objekt an das \verb|GetObjectPresignRequest| Objekt übergeben und die Gültigkeitsdauer der Signatur angegeben. Die Gültigkeitsdauer bestimmt, wie lange die signierte URL gültig sein wird. Um die signierte URL zu generieren, wird das \verb|GetObjectPresignRequest| Objekt an den \verb|S3Presigner| übergeben, auf den die Methode \verb|presignGetObject()| aufgerufen wird. Das generierte \verb|PresignedGetObjectRequest| wird dann in Zeile 19 als String gespeichert und ausgegeben.

\newpage

Bei der Methode von Cloud Storage ist der Vorgang zur Erstellung des signierten URLs kürzer. Folgende Abbildung zeigt die Methode zur Generierung des signierten URLs für Cloud Storage:

\begin{lstlisting}[language=Java, caption=Prototyp Code Snippet - Generierung eines signierten URLs durch GC]
@Override
public void getPresignedUrl(String bucketName, String key, Integer minutes, String encryptionKey) {

    Map<String, String> kmsKeyName = new HashMap<>();
    kmsKeyName.put("kmsKeyName", encryptionKey);

    BlobInfo blobInfo = BlobInfo.newBuilder(BlobId.of(bucketName, key))
            .setMetadata(kmsKeyName)
            .build();
    
    URL url =
            this.storage.signUrl(blobInfo, minutes, TimeUnit.MINUTES, Storage.SignUrlOption.withV4Signature());
    
    System.out.println("Generated GET signed URL: " + url);
}
\end{lstlisting}

Ähnlich wie bei S3 wird eine Referenz zum Bucket erstellt, indem man den Bucket Namen und den Namen des Objekts dem \verb|BlobInfo| Objekt mitgibt. Anschließend kann die URL durch Aufrufen der Methode in Zeile 11-12 erstellt werden. Hier wird das \verb|BlobInfo| Objekt, die Minuten in und die Signatur Methode übergeben. Zuletzt wird die URL in der Kommandozeile ausgegeben.\\

Mit signierten URLs können die Anforderungen von leoticket an die sichere Speicherung und Bereitstellung der Daten über signierte URLs berücksichtigt werden. Dies bedeutet, Kunden können über diese Links die Dateien für Tickets und Rechnungen herunterladen ohne das Problem von zu großen Email-Anhängen zu haben.\\

\newpage

\section{Messung der Performance}

In diesem Abschnitt erfolgt die Messung der Performance für die Dienste von AWS und GC. Dabei werden bis zu 1000 generierte Dateien sowohl für den Upload als auch für den Download betrachtet. Die Performance Analyse wird auf einer virtuellen Maschine aus Hetzner ausgeführt, um eine realistische Messung zu gewährleisten. Die Performance-Messung beim Hoch- und Herunterladen kann jedoch von verschiedenen Faktoren wie Netzwerklatenz, verfügbarer Bandbreite, Serverkapazität, Datenmenge und der Auslastung der Server beeinflusst werden. Außerdem kann der Hetzner Server näher an AWS oder GC liegen und könnte dazu führen, dass dabei die Dauer des Hoch-, und Herunterladens aus diesem Grund schneller sein kann. Die Messungen dienen lediglich des groben Vergleichs beider Cloud Provider. Die Performance-Messung wird auf verschiedene Speicherklassen durchgeführt, um einen Vergleich zwischen dieser zu ermöglichen und eine Grundlage für die Bewertung ihrer Leistungsfähigkeit zu schaffen.\\

\textbf{AWS}\\

AWS bietet Dienste für die Performance Analyse. Unter anderem die Amazon CloudWatch, S3 Storage Lens und die S3 Transfer Acceleration. Die AWS CLI stellt einfache Methoden zum S3 Upload und Download Tests vor. Beispielsweise kann man mit dem Befehl:

\begin{lstlisting}
	aws s3 cp <lokaler_pfad> s3://<Bucket_Name>/<Ziel_Dateipfad>
\end{lstlisting}

Dateien hoch,-und herunterladen und die Zeit für die benötigte Request messen. Auch mit der AWS SDK können Performance Test Skripte geschrieben werden. Diese Methode wird für den Prototypen angewendet. Dabei werden Tests bereitgestellt, die mehrere Dateien automatisch hoch, und herunterladen und dabei die Zeit messen, die vergangen ist.\\ 

\textbf{GC Storage}\\

GC bietet einen eigenen \verb|gsutil| Tool für die Performance Analyse. Im Abschnitt Performance bereits erwähnt können durch die \verb|perfdiag| Funktion Performance Diagnosen erstellt werden. 

\begin{quote}
	Mehrere Testdateien werden aus einem angegebenen Bucket hoch-und heruntergeladen. Nach der Analyse werden alle Testdateien wieder gelöscht nach erfolgreicher Diagnose. Die \verb|gsutil| Performance kann von einigen Faktoren beeinflusst werden wie vom Client, Server oder Netzwerk. Möglich sind die CPU Dauer, der verfügbare Speicher, die Netzwerk Bandbreite, Firewalls und Fehlerraten zwischen \verb|gsutil| und den Google Servern. Die \verb|perfiag| Funktion wurde dafür bereitgestellt, damit Nutzer Messungen durchführen können, die beim Troubleshooting von Performance Problemen helfen, vgl. \cite{gc-perfdiag}.
\end{quote}

Um die Performance Diagnose auszuführen, kann der folgende Befehl verwendet werden:

\begin{lstlisting}
	gsutil perfdiag -o test.json -n 67000 -s 400kb gs://leoticket-bucket
\end{lstlisting}

Die \verb|-o| Option schreibt den Output des Ergebnisses in eine Datei. Die Output Datei ist eine JSON Datei mit System Informationen und enthält die Performance Diagnose Ergebnisse. Die -n Option setzt die Anzahl der Objekte, die heruntergeladen und hochgeladen werden sollen während dem Test. Mit der \verb|-s| Option kann man die Objektgröße in bytes angeben. Zum Schluss wird der Name des Buckets angegeben. Damit der Befehl erfolgreich ausgeführt werden kann, braucht man die entsprechenden Rechte und muss sich authentifizieren können.\\

Um die Performance der SDKs zu analysieren, werden Objekte mit jeweils 100kb Objektgröße in verschiedenen Speicherklassen hoch-, und heruntergeladen. Anschließend wird die Performance über die SDK von Cloud Storage ähnlich wie bei AWS getestet. Der Test wird auf einer virtuellen Maschine von Hetzner ausgeführt.\\

Die Performance wird schrittweise gemessen, beginnend mit einer Datei bis hin zu 1000 Dateien in Zehner-Schritten. Dies bedeutet, dass Messungen für eine Datei, zehn Dateien, 100 Dateien und 1000 Dateien durchgeführt werden. Zur Durchführung der Messung wird eine Java Jar-Datei des Prototyps erstellt, in der die Testmethoden ausgeführt werden. Diese Testmethoden generieren zunächst Testdaten, die mit zufälligen String-Werten der Größe 100 KB gefüllt sind. Anschließend werden die Testmethoden in der Kommandozeile ausgeführt.

\newpage

\subsection{Messungsergebnisse}

In diesem Abschnitt werden die Messungsergebnisse der Performance Messung in genauen Millisekunden-Zahlen präsentiert. Die Speicherklassen Standard, Standard-IA und One Zone-IA von AWS wurden jeweils mit Standard, Nearline und Coldline von GC gemessen und verglichen. Im folgenden werden die Ergebnisse der Upload und Download Dauer aller Speicherklassen präsentiert: 

\begin{code}
STANDARD-IA vs. NEARLINE

1 File---------------------------------------------------------------------------

Elapsed Time for 1  Object Upload in S3:-------------------------------705ms.
Elapsed Time for 1 Object Upload in Cloud Storage:---------------------841ms.

Elapsed Time for 1 Object Download in S3:------------------------------18ms.
Elapsed Time for 1 Object Download in Cloud Storage:-------------------26ms.

10 Files-------------------------------------------------------------------------

Elapsed Time for 10  Object Uploads in S3:-----------------------------1862ms.
Elapsed Time for 10 Object Uploads in Cloud Storage:-------------------3217ms.

Elapsed Time for 10 Object Downloads in S3:-----------------------------35ms.
Elapsed Time for 10 Object Downloads in Cloud Storage:------------------94ms.

100 Files------------------------------------------------------------------------

Elapsed Time for 100  Object Uploads in S3:-----------------------------16681ms.
Elapsed Time for 100 Object Uploads in Cloud Storage:-------------------22683ms.

Elapsed Time for 100 Object Downloads in S3:----------------------------129ms.
Elapsed Time for 100 Object Downloads in Cloud Storage:-----------------269ms.

1000 Files-----------------------------------------------------------------------

Elapsed Time for 1000  Object Uploads in S3:----------------------------71175ms.
Elapsed Time for 1000 Object Uploads in Cloud Storage:------------------1851348ms.

Elapsed Time for 1000 Object Downloads in S3:---------------------------663ms.
Elapsed Time for 1000 Object Downloads in Cloud Storage:----------------1489ms.
\end{code}

\newpage

\begin{code}
STANDARD vs. STANDARD

1 File---------------------------------------------------------------------------

Elapsed Time for 1  Object Upload in S3:--------------------------------585ms.
Elapsed Time for 1 Object Upload in Cloud Storage:----------------------660ms.

Elapsed Time for 1 Object Download in S3:-------------------------------28ms.
Elapsed Time for 1 Object Download in Cloud Storage:--------------------19ms.

10 Files-------------------------------------------------------------------------

Elapsed Time for 10  Object Uploads in S3:------------------------------1369ms.
Elapsed Time for 10 Object Uploads in Cloud Storage:--------------------3002ms.

Elapsed Time for 10 Object Downloads in S3:-----------------------------70ms.
Elapsed Time for 10 Object Downloads in Cloud Storage:------------------108ms.

100 Files------------------------------------------------------------------------

Elapsed Time for 100  Object Uploads in S3:-----------------------------8110ms.
Elapsed Time for 100 Object Uploads in Cloud Storage:-------------------20750ms.

Elapsed Time for 100 Object Downloads in S3:----------------------------180ms.
Elapsed Time for 100 Object Downloads in Cloud Storage:-----------------396ms.

1000 Files-----------------------------------------------------------------------

Elapsed Time for 1000  Object Uploads in S3:----------------------------73992ms.
Elapsed Time for 1000 Object Uploads in Cloud Storage:------------------176004ms.

Elapsed Time for 1000 Object Downloads in S3:---------------------------650ms.
Elapsed Time for 1000 Object Downloads in Cloud Storage:----------------1619ms.

\end{code}

\newpage

\begin{code}
ONEZONE-IA vs. COLDLINE

1 File---------------------------------------------------------------------------

Elapsed Time for 1  Object Upload in S3:--------------------------------505ms.
Elapsed Time for 1 Object Upload in Cloud Storage:----------------------636ms.

Elapsed Time for 1 Object Download in S3:------------------------------28ms.
Elapsed Time for 1 Object Download in Cloud Storage:-------------------18ms.

10 Files-------------------------------------------------------------------------

Elapsed Time for 10  Object Uploads in S3:------------------------------1105ms.
Elapsed Time for 10 Object Uploads in Cloud Storage:--------------------2855ms.

Elapsed Time for 10 Object Downloads in S3:-----------------------------65ms.
Elapsed Time for 10 Object Downloads in Cloud Storage:------------------89ms.

100 Files------------------------------------------------------------------------

Elapsed Time for 100  Object Uploads in S3:-----------------------------7539ms.
Elapsed Time for 100 Object Uploads in Cloud Storage:-------------------20391ms.

Elapsed Time for 100 Object Downloads in S3:----------------------------187ms.
Elapsed Time for 100 Object Downloads in Cloud Storage:-----------------250ms.

1000 Files-----------------------------------------------------------------------

Elapsed Time for 1000  Object Uploads in S3:----------------------------73086ms.
Elapsed Time for 1000 Object Uploads in Cloud Storage:------------------163661ms.

Elapsed Time for 1000 Object Downloads in S3:---------------------------701ms.
Elapsed Time for 1000 Object Downloads in Cloud Storage:----------------1559ms.
\end{code}

Die Ergebnisse werden im Kapitel Zusammenfassung der Ergebnisse untersucht und als Liniendiagramm dargestellt.

\newpage

\section{Zusammenfassung der Implementierung}

Der Prototyp soll einen Vergleich zwischen den beiden Cloud Providern bieten. Das Ziel dabei ist es, ähnliche Technologien von beiden Providern zu verwenden und die Performance mit Testdaten dabei messen zu können. 

Der Prototyp implementiert zwei wichtige Funktionen: Das Hochladen und Herunterladen von Dateien unter Berücksichtigung der Anforderungen von leoticket. Dabei werden signierte URLs zur Bereitstellung der Dateien verwendet und die SSE KMS von beiden Providern für die Datenverschlüsselung angewendet. Der Prototyp soll auch als externe Bibliothek für eigene Anwendungen integriert werden können. Bei der Implementierung wurden die offiziellen Dokumentationen von AWS und Google Cloud verwendet, um die aktuellsten Versionen zum Zeitpunkt der Arbeit zu verwenden. Auf Wunsch von Leomedia wurde Spring Boot als Framework gewählt, um eine Enterprise-Anwendung bereitzustellen. Java wurde aus Präferenzgründen als Programmiersprache gewählt, obwohl beide Provider viele weitere Sprachen, die bereits im Kapitel zur API-Anbindung erwähnt wurden, unterstützen.\\

Der Prototyp wurde so aufgebaut, dass zwischen AWS und GC durch eine Umgebungsvariable gewählt werden kann. Die Klasse \verb|CloudStorageServiceFactory| stellt die Auswahl zwischen AWS und GC zur Verfügung. Je nachdem welchen Cloud Provider der Nutzer angegeben hat, wird die entsprechende Klasse instanziert und aufgerufen. Die Klassen \verb|AWSS3StorageService| und \verb|GoogleCloudStorageService| implementieren die Methoden des Interfaces \verb|CloudStorageService|.\\

Die bereitgestellten Tests dienen der Performance Messung. Dabei werden Dateien automatisch als 100kb große Objekte erstellt und durch die Testmethoden verwendet, um Objekte hoch-, und herunterzuladen. Sie unterstützen dabei, die Dauer der verschiedenen Funktionen zu messen.\\

Da die Maven Abhängigkeiten kein Teil der Spring Cloud Release Train sind, hängt es von der Community ab, die Versionen aktuell zu halten. Für AWS bedeutet dies, dass die AWS SDK 2.0 Version nicht komplett abgedeckt ist. Jedoch ist sie so weit, dass S3 bereits unterstützt wird. 
\chapter{Ergebnisse dieser Arbeit}

\section{Beschreibung und Funktionalität des Prototyps}

\section{Zusammenfassung der Ergebnisse}

\subsection{Kalkulationsergebnisse}

\subsection{Messungsergebnisse}
\chapter{Diskussion}

In diesem Kapitel werden die Eigenschaften, der Prototyp, Kalkulations-, und die Messungsergebnisse der Performance analysiert und interpretiert. Dabei werden zuerst die Kalkulationsergebnisse und Messungsergebnisse untersucht. Anschließend werden die Eigenschaften und der Prototyp bewertet und dabei auf die Grenzen des Prototyps eingegangen.
  
\section{Analyse und Interpretation der Ergebnisse}

Aus der Kostenanalyse und den Kalkulationsergebnissen hat sich ergeben, dass Google Cloud insgesamt niedrigere Kosten als AWS bietet. 

\section{Bewertung des Prototyps}

Stärken und Schwächen
\chapter{Fazit}

In diesem Kapitel werden auf die gestellten Fragen zurückgegriffen und beantwortet, sodass der rote Faden der Arbeit nochmal deutlich wird. Anschließend wird die potenzielle Anwendung des Prototyps diskutiert.

\section{Beantwortung der Forschungsfrage}

Um die gestellten Fragen zu beantworten, werden sie zunächst aufgegriffen. Folgenden Fragen wurden am Anfang der Arbeit gestellt:

\begin{itemize}
	\item Welches Speichersystem ist im Hinblick auf Kosten, Performance und Verfügbarkeit für die Persistenz von Binärdaten besonders geeignet? 
	\item Wie können Daten durch sichere, zeitlich begrenzte URL's bereitgestellt werden?
\end{itemize}

Um die erste Frage zu beantworten, werden die Punkte vom theoretischen Teil aufgegriffen. Es wurden verschiedene Speicherarten wie Objekt-, Block-, und File Storage untersucht. Dabei stellte sich heraus, dass Objekt Storage als Speichersystem für die Anforderungen von leoticket geeignet ist. Einige Cloud Provider stellen Objekt Storage zur Speicherung von Daten zur Verfügung und ist im aktuellen Markt stark vertreten. Es wurden die zwei größten Cloud Provider AWS und GCP betrachtet und eine Vergleichsbasis hergestellt in Hinblick auf Kosten, Performance, Verfügbarkeit, Sicherheit, Bereitstellung der Daten und API Anbindungen. Bei der sicheren Speicherung war es wichtig, dass die Daten vertraulich gespeichert werden und nur berechtigte Zugriff auf die Daten haben.\\

 Datenverschlüsselungsmethoden wurden bei beiden Cloud Providern betrachtet und dabei festgestellt, dass die SSE C zwar die stärkste, unabhängige Sicherheit bietet, jedoch das Risiko besteht, selbstverwaltete und gespeicherte Schlüssel zu verlieren. Außerdem müssten Mitarbeiter dafür geschult werden, was extra Aufwand bedeutet. Aus diesem Grund wurde für die Implementierung des Prototyps die SSE-KMS customer-managed Methode verwendet, damit der Nutzer die Schlüssel in der KMS von den Providern selber erstellen und verwalten kann. So bleibt die Kontrolle noch und verschafft höhere Sicherheit. Bei der Verfügbarkeit versprechen beide Provider eine Verfügbarkeit von 99.9\% und hängt von den Speicherklassen ab, für die man sich entscheidet. Bei der Untersuchung der Speicherklassen in Verbindung mit Kosten und Performance stellte sich heraus, dass die Standard-IA von AWS und die Nearline von GC besser zu den Anforderungen von leoticket passt. Da die Latenz für leoticket kein Kriterium darstellt und vernachlässigt werden kann, fallen die Standard Klassen beider Provider weg. Die Speicherung der Daten steht mehr im Fokus, da auf die Daten maximal 2 mal zugegriffen werden. Die OneZone-IA fällt auch weg, da die Daten in nur einer Availability Zone gespeichert wird. Das Risiko für Datenverlust durch der Nicht-Verfügbarkeit dieser Zone steigt dadurch. Daten müssen auf Abruf schnell zugreifbar sein, deshalb sind die OneZone-IA und die Coldline nicht geeignet, da sie eher für selten abgerufen Daten angepasst sind. Die Abrufkosten dieser Speicherklassen sind am höchsten und nicht zu empfehlen.\\

Insgesamt wird für die Persistenz von Binärdaten ein Object Storage mit den Speicherklassen Standard-IA von AWS und die Nearline von GC empfohlen. Sie bieten die nötigen Funktionen an, um die Anforderungen zu decken und kostengünstig Daten auf längerer Zeit zu speichern und dabei eine hohe Verfügbarkeit und eine schnelle Performance zu berücksichtigen. Die Präferenzen hängen auch mit der Entscheidung ab, was das Unternehmen braucht und präferiert. Beide Provider bieten eine gute Objektspeicherung kostengünstig und leistungsfähig an.\\

Bei der zweiten gestellten Frage geht es um die Bereitstellung der Daten durch signierte zeitlich begrenzte URLs. Diese Frage wurde durch Ausprobieren der SDKs beim Prototypen untersucht und bewertet. Es stellt sich heraus, dass beide Cloud Provider die Funktionen anbieten, signierte URLs zu erstellen und zeitlich bereitzustellen. Durch den Prototypen kann man Dateien hochladen und sie durch diese URLs bereitstellen. Durch Klicken auf den generierten Link werden die Daten heruntergeladen. So kann verhindert werden, Dateien nicht direkt in Email Anhängen hinzuzufügen und diese durch die Links bereitstellen zu können. Diese Dateien werden von den Buckets entschlüsselt heruntergeladen und Nutzer können ohne AWS oder GC Credentials darauf zugreifen. Diese URLs sind zeitlich begrenzt. Über den Prototypen kann man die Minuten, in der der Link valide ist, angeben. GC und AWS stellen ausführliche Dokumentationen auf den offiziellen Seiten bereit, um diese Funktionen zu implementieren und anzuwenden in verschiedenen Programmiersprachen.\\

So kann leoticket vom alten System zu der neuen empfohlenen Speicherlösung wechseln, um Daten sicher und schnell bereitzustellen und auf längerer Zeit zu speichern. Es ist zu beachten, dass diese Arbeit lediglich dem Vergleich und der Veranschaulichung beider Cloud Provider dient und das jedes Unternehmen unterschiedliche Anforderungen aufweist. Diese Arbeit dient als Stütze und zum Ausprobieren der Technologien durch den Prototypen. 


\section{Potenzielle Anwendung des Prototyps}

Der fertige Prototyp kann als Bibliothek und Referenz für eigene Anwendungen angewendet werden. 
Der Prototyp dient als Stütze und kann getestet werden für Zwecke wie das Hochladen und Herunterladen von Objekten. Außerdem kann der Prototyp die benötigten Buckets mit den Konfigurationseinstellungen wie Lifecycle Rules, Object Versioning usw. durch Terraform erstellen.
\include{./Ausblick/Ausblick}

%%%%%%%%%%%%%%%%%%%%%%%%%%%%%%%%%%%%%%%%%%%%%%%%%%%%%%%%%%%%%%%%%%
% Literaturverzeichnis ausgeben
%%%%%%%%%%%%%%%%%%%%%%%%%%%%%%%%%%%%%%%%%%%%%%%%%%%%%%%%%%%%%%%%%%
\chapter{Literaturverzeichnis}
\markboth{Literaturverzeichnis}{Literaturverzeichnis}
\printbibliography[heading=literatur,keyword=literatur]
\clearpage
\printbibliography[heading=pdf,keyword=pdf]
\clearpage
\printbibliography[heading=online,keyword=online]
\clearpage
\phantomsection

% Anhang

\chapter{Anhang}

\section{Repositories}
\subsection{Github Link}

\subsubsection*{Prototyp:}

HTTPS: \url{https://github.com/gmzbae/cloud-gcs-aws3.git}

\begin{verbatim}SSH: git clone git@github.com:gmzbae/cloud-gcs-aws3.git\end{verbatim}			

\subsubsection*{Thesis:}

HTTPS: \url{https://github.com/gmzbae/bachelor-thesis-latex.git}

\begin{verbatim}SSH: git clone git@github.com:gmzbae/bachelor-thesis-latex.git \end{verbatim}

\subsection{Dokumentation}

Code Dokumentation\\
Entwickler Handbuch

\subsection{Code Snippets}

\begin{lstlisting}
	
\end{lstlisting}

\clearpage

\end{document}






